\chapter{Conclusion}
The potential of open web can only be harnessed to its full extent by adding structure to it. The process involves 
creating structured repositories derived from the web that can answer interesting questions pertaining to entities
that exist on the web. 

Many such smart applications that rely on structured web will rely on frequencies of occurrence of the former. 
The report has been a buildup to achieving that.
We started by briefing what knowledge bases are. In the second part, we introduced the problem of disambiguating the
mentions of named entities and presented solutions roughly spanning last 8 years of research in the field. 

In the third part, we elaborated on what is meant by aggregate statistics and presented several applications of the same. 
We presented maximum mean discrepancy approach for class ratio estimation via an example and discussed the problem formulation. We briefly outlined how mmd can be applied for
 estimating occurrence statistics of entities.

State of the art approaches for named entity disambiguation brush the figure of 90\% accuracy. It is thus expected
that the focus of the community will now shift to making the process of disambiguation faster and integrating the 
disambiguators in the search pipeline. It remains to be seen how approaches based on direct estimation of entity occurrence 
ratios perform in comparison with the standard tools, both in terms of speed and accuracy. 