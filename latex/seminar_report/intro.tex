\chapter{Introduction}
\section{Problem Statement}
Given a knowledge base such as Yago or Freebase consisting of entities and relations, and the Web, our goal is 
to attach reliable estimates of the frequency of occurrences on the Web of various entities and relations as 
singletons, pairs (ordered and unordered) in a sentence.  
The aim is to collect statistics so as to be able to design prior probabilities about the set of 
entities and relations that can co-exist in a sentence or a paragraph. These statistics have applications in 
query interpretation, language understanding tasks.  
We can view it as being analogous to statistics in relational catalogs.

\section{Structure}
The report is divided into three parts.

\textbf{The first part} gives an overview of what are knowledge bases. This is important 
since the concept of such repositories of structured knowledge is central to the
report. 

\textbf{The second part} begins with an introduction to the problem of named entity disambiguation, the 
terminology and applications, and goes on to cover the techniques for named entity disambiguation
in some detail. We give and overview of the two broad categories of disambiguation techniques, Local and
global disambiguation.

\textbf{The third part} begins with a discussion on definition of Aggregate statistics and 
some of their applications. Finally, we discuss Maximum mean discrepancy and its 
application for estimating the aggregate statistics over entities.