%sg
\documentclass[a4paper,10pt]{report}
\usepackage[utf8]{inputenc}
\usepackage{color}
\usepackage{url}
\usepackage{amsmath}
\usepackage{mathtools}
\usepackage{epigraph}
\usepackage{graphicx, float}
\usepackage{pst-node}
\usepackage[]{algorithm2e}
\newtheorem{mydef}{Definition}
% Title Page
\title{Seminar Report\footnote{Draft}}
\author{Aman Madaan \\ 133050004}


\begin{document}
\maketitle
\tableofcontents

\begin{abstract}
TODO
\end{abstract}

\part{Problem Definition and Motivation}
\section{Named Entity Disambiguation}
\subsection{Problem Definition}
Consider the following sentence : 

 \begin{center}
\textcolor{blue}{Michael Jordan is a Professor at Berkeley}
   \end{center}

 We first want to identify all the \textbf{named entities} in the text. The task is called named entity recognition and is 
 formally defined as : 
 \begin{mydef}[Named entity recognition\footnote {from \ref{thewiki}}]
 \label{nerdef}
   Named-entity recognition (NER) (also known as entity identification and entity extraction) is a subtask of information extraction that seeks to locate and classify 
   atomic elements in text into predefined categories such as the names of persons, organizations, locations, expressions of times, quantities, monetary values, percentages, etc.
  \end{mydef}
 But we do not stop at that, we want to link each of the named entities thus recognized to a knowledge base\footnote{The knowledge base is a catalog of entities, like Wikipedia}.
 Thus, our problem has a 2 step solution : 

 \begin{itemize}  
  \item Step 1 : \textbf{Identify} entities
  \medskip
  
  \textcolor{green}{Michael Jordan\_PERSON} is a professor at \textcolor{green}{Berkeley\_INSTITUTION} \medskip
  \item Step 2 : \textbf{Link} entities to knowledge bases : 
  \medskip
  
  \textcolor{red}{Michael Jordan\_ENTITY} (\url{http://en.wikipedia.org/wiki/Michael_I._Jordan})  is a professor at  
  \textcolor{red}{Berkeley\_ENTITY} (\url{http://en.wikipedia.org/wiki/University_of_California,_Berkeley})
\end{itemize}

The stanford NER library is a popular choice for recognizing named entities. [\ref{stanfordner}]

\subsection{Applications}


In simple terms, disambiguating named entities in the unstructured text imparts a structure to the document. 
We need two more data points to further appreciate the power that such a tool provides to us.
The first is the size of the web. As of 31st March 2014, there are atleast 1.8 billon indexed web pages.[\ref{ws}]
The second is the number of wikipedia entities. The wikipedia statistics [\ref{wikistats}] estimate the number of pages to be
around 32 million. Yago, a catalog of entities made from wikipedia has 12, 727, 222 entities.	
Imparting structure to documents at this magnitude has far reaching implications in the information
extraction and is a bridge towards the hitherto dream of a semantic web.  \\


It is highly recommended that the reader has a look at \url{http://www.google.co.in/insidesearch/features/search/knowledge.html}, 
The google knowledge graph project.


\subsection{Terminology}
The following terms are widely used in the literature on named entity disambiguation and thus in the survey.

\begin{itemize}
 \item \textbf{Mention, Spot} \\
 A piece of text which needs to be disambiguated. For example, \textbf{Amazon} has attracted a lot of visitors.
 \item \textbf{Entity} \\
 A named entity as defined in the definition \ref{nerdef}. 
 \item \textbf{Candidates} \\
 A set of entities which might be the correct disambiguation for a given mention.
 For example, possible candidates for the sentence above are Amazon river and Amazon.com
 \item \textbf{Prior} \\
 Probability of a mention linking to a particular entity. For example, the mention ``Amazon`` may be used
 to refer to the website (say) 60\% of the time.
 \item \textbf{Knowledge base} \\
 A catalog of Entities where an entity is as defined above. For example, Wikipedia or yago.

\end{itemize}

\subsection{Structure}
We have already given an introduction to the problem and the applications. 
The next section discusses the solutions based on local disambiguation, i.e., figuring 
out the correct entity based on just the local evidences. Chapter 3 presents the intuition
behind having a global strategy for disambiguation, and the optimization problem that
results from such an objective. The final section summarizes a recent work which 
pragmatically selects global and local evidences, to get the best of both worlds.


\part{Structured Knowledge Repositories}

\begin{frame}{Knowledge Bases}
\begin{itemize}
\item Want to make the web smarter by \emph{understanding} the content \medskip
\item \emph{What} Entities? \emph{Which} Relations? \medskip
\item \emph{Encyclopedia that a Computer can understand} \medskip
 \item A standard reference set of entities, relations, type hierarchies \medskip
 \item \textbf{Wordnet}  The maiden knowledge base, has clean type system but limited entity base \medskip
 \item \textbf{Wikipedia} Huge, crowd sourced, but extremely loose and vague type systems \medskip
 \item Several knowledge bases have emerged as  middle ground 
\end{itemize}
\end{frame}

\begin{frame}{Freebase}
\begin{figure}[h]
 \centering
 \includegraphics[bb=0 0 1366 768,scale=0.25]{./freebase.png}
 % freebase.png: 1366x768 pixel, 72dpi, 48.19x27.09 cm, bb=0 0 1366 768
\end{figure}
 
\end{frame}
\begin{frame}{Freebase}
\begin{itemize}
 \item Freebase relies on crowd sourcing for creation of a rich but clean knowledge base \medskip
 \item The development of Freebase follows the same chain as 
Wikipedia, with users flagging issues, and cleaning and augmenting information \medskip
\item Freebase also provides access to itself using web APIs.
\end{itemize}
\end{frame}

\begin{frame}{Knowledge Bases : Not there yet}
 \begin{itemize}
  \item \textcolor{red}{None} of the knowledge bases provides \textcolor{red}{entity priors} of any kind \medskip
  \item \textcolor{red}{Co-occurrence statistics} are also missing \medskip
  \item These pieces of information are really crucial for a number of tasks related like querying knowledge
  graphs. \medskip
  \item We motivate the need for such statistics after reviewing named entity disambiguation techniques. 
 \end{itemize}

\end{frame}



\part{Named Entity Disambiguation}
\chapter{Local Disambiguation of named entities}
\section{Introduction}
In local disambiguation, we collect just local evidences for each 
mention for its disambiguation. This was state of the art until the CSAW[\ref{thepaper}]
paper came along. We start by defining the problem and discussing the general form of solutions.
We then provide a short summary of approach followed in Wikify [\ref{wikify}] and the famous Milne and Witten paper [\ref{mw}]. A
solution based on machine learning[\ref{thepaper}] concludes the chapter.

\section{Problem definition}
We need to disambiguate a mention by collecting the local evidences. 
The evidences can be anything, POS tags, gender information, dictionary lookup 
etc. By local disambiguation, we mean that \textbf{we cannot use the disambiguation
information for any other entities for solving the problem.} 

\section{Solutions}

Every local disambiguation techniques fall into one of the following two categories[\ref{wikify}]

\begin{itemize}
 \item \textbf{Knowledge based} \\
 Derived from the classical word sense disambiguation literature, this 
 technique depends on the information drawn from the definitions provided by the knowledge base. 
 (See Lesk's algorithm [\ref{lesk}]).
 This is based on the overlap of context with the definitions of each of the candidate 
 senses as given in the knowledge base.
 
 \item \textbf{Machine Learning based} \\
 This method is based on collecting features from the mention and its surroundings, and
 training a classifier to give a verdict on a particular sense being a likely disambiguation
 of a mention. Machine learning based local disambiguation was almost unanimously adopted
 by the ned community as the solution for local disambiguation. AIDA changed the scene 
 by introducing a knowledge based local similarity score which works well.
 
 \end{itemize}

 \section{Related Work}

\subsection{Wikify[\ref{wikify}]}
The biggest contribution of this paper is perhaps presenting Wikipedia as the 
catalog against which were supposed to disambiguate. The paper also identifies
two broad methods of doing named entity disambiguation : Knowledge based and 
data based. Since the paper dates back to 2007, when the problem of NED was 
not as established, there are a lot of references to the problem of word disambiguation.

\subsection{Learning to link with Wikipedia[\ref{mw}]}
This paper defined three different features for disambiguation : 
\begin{itemize}
 \item Commonness  : This is the prior defined in Chapter 1.
 \item Relatedness : Perhaps the biggest contribution of this paper, the relatedness score,
 gives a measure for determining how similar the two entities are. This measure 
 is based on the number of common inlinks to entities in question.
 The relatedness measure as defined here has been used in a lot of works. In fact, all
 the approaches presented in the subsequent sections use this relatedness score, popular
 as the Milne-Witten score for finding out entity entity similarity.
 This score is defined as follows \\ \\
 $ r(\gamma, \gamma') = \frac{log|g(\gamma) \bigcap g(\gamma')| - log(max\{|g(\gamma)|, |(\gamma')|\})} {log c - log(min\{|g(\gamma)|, |(\gamma')|\})}$ 
 
 Where 
   \begin{itemize}
    \item $g(\gamma)$ : Set of wikipedia pages that link to $\gamma$
    \item $c :$ Total number of Wikipedia pages
    \item $r(\gamma, \gamma') :$ Relatedness of topics $\gamma$ and $\gamma'$
   \end{itemize}\bigskip

 
\end{itemize}
The algorithm selects a few unambiguous links in the document, and uses the similarity of the candidates
with these unambiguous links as a criteria for disambiguation.
Thus, in some sense, although the technique is not totally local, it shies away from doing anything to maintain
coherence among the entities that are unveiled and thus we do not call this method a ``Global method'', which 
are discussed in the following chapter.

\section{Machine learning based local disambiguation}
As mentioned, there are primarily two approaches for local disambiguation.
This section discusses a machine learning based local disambiguation method in some detail. This section
is based on the local disambiguation approach taken in [\ref{thepaper}].
\subsection{Definitions}
We first repeat the definitions for quick reference : 
\begin{itemize}
  \item $s$ : Spot, an Entity to be disambiguated (Christian leader John Paul) \bigskip 
  \item $\gamma$ : An entity label value (\url{http://en.wikipedia.org/wiki/Po-pe_John_Paul_II})  \bigskip 
 \item $f_s(\gamma)$ : A feature function that creates a vector of features given a spot and a candidate entity label.
 \end{itemize}
 
 \subsection{Local compatibility : Feature design} 
 The feature function takes the spot and the candidate as arguments. 
 
\begin{itemize} 
 
 \item The following information about a candidate $\gamma$ is used
\begin{itemize} 
 \item Text from the first descriptive paragraph of $\gamma$
  \item Text from the whole page for $\gamma$
  \item Anchor text within Wikipedia for $\gamma$.
  \item Anchor text and 5 tokens around $\gamma$ 
 \end{itemize}
 
 \item We now have 4 pieces of information about $\gamma$. We take each of these, and apply the following operations with 
 one argument as the spot
    \begin{itemize}
      \item{Dot-product between word count vectors}
      \item{Cosine similarity in TFIDF vector space}
      \item{Jaccard similarity between word sets}
  \end{itemize} 
  \end{itemize} 
 
 Thus, for a candidate - mention pair, we get a total of 12 Features (3 operations, 4 argument pairs). 
 
 In addition to these, we also use a sense probability prior as defined in the introduction. A popular way of 
 obtaining the prior is counting the number of times the spot has been linked to a particular entity. For example, 
 the hypertext ``Linux'' might be linked to the page for the Linux kernel 70\% of the times, and to the page
 for Linux based operating systems rest of the times.
 
 
\subsection{Compatibility Score}
Once we have the features, we train the classifier by using the following optimization objective : 
\begin{itemize}
 \item Local compatibility score between a spot $s$ and a candidate is given by $w^{T}f_s(\gamma)$
 \item $w$ is trained using an SVM like training objective
 \begin{center} $w^{T}f_s(\gamma) - w^{T}f_s(\gamma) \geq 1 - \epsilon_s$ \end{center}
 \end{itemize}
 
 \subsection{Finding the best candidate}

 \begin{algorithm}[H]
 \KwData{A Document d}
 \KwResult{Annotated document d' with every mention linked to the best candidate entity}
 \ForEach{mention $m$ in the document} {
  calculate $argmax_{c_m \in \Gamma}w^{T}f_m(c_m)$
  where $\Gamma$ = $c_m$ : $c_m$ is a possible disambiguation of $m$
 }\caption{Local disambiguation}
\end{algorithm}

 Note that a multi class classifier is not learned for several reasons, all of which can be mapped to 
 the large number of classes. 
\input{ned.tex}

\part{Aggregate Statistics}
\chapter{Distributional Statistics of Named entities}
Once you have a catalog of things, it makes sense to ask which of these ``things'' are more important than the others.
In fact, one might extend the question and ask, ``Which pairs (or triples) of these things appear together on the open web?''.
We define several different statistics one might be interested in over these entity catalogs, discuss some applications, 
propose a baseline method and finally, prepare the ground for the next chapter by giving an outline of a solution 
which is aimed at directly providing us with the statistics we are looking for. 

\section{What Statistics?}

\subsection{Which sense dominates for an entity?}
For starters, we might want to calculate the number of times a particular ``sense'' of an entity\footnote{Please note that we 
refer to entity in general terms. For example, any object having a YAGO id is an entity}. 
For example, the ``Gingerbread'' might refer to several different concepts; from perhaps the most famous Android 2.3 to the novel.
We want to know out of all the \emph{Gingerbreads} on the web, how many refer to the Operating system. 
We call this number the sense prior.
\begin{equation}
\tag{1}
\text{Sense Prior}(S_i, E) =  P(E\text{ appears as the $i^{th}$ sense}) = P(S_i | ``E'')
\end{equation}

Where $S_i$ is the $i^{th}$ sense\footnote{$i^{th}$ disambiguation in Wikipedia parlance} of the entity E. 

\subsection{How often do the 2 entities appear together?}
A second interesting statistic would be to count how many times do two given entities, taking two given senses appear together.
For example, We might want to know how many times does Nokia \url{http://en.wikipedia.org/wiki/Nokia} appears with Gingerbread \url{http://en.wikipedia.org/wiki/Gingerbread_(operating_system)}

We call should counts Entity bi grams. We note that in contrast to word bi-grams and relational grams [\ref{relgram}], entity bi grams
are symmetric, and there is no obvious use case where we might need to know the order dependent occurrence count of the entities. 
However, such a formulation will lead to a sparse distribution, since each count will have to be normalized by the total number of 
entity bigrams. We thus define the entity bi gram count as follows : 
\begin{equation}
 \tag{2}
 \text{Entity Bi Gram}(E2 | E1) = P(E2\text{ follows }E1) = P(E2 | E1) 
\end{equation}

We propose an application of Entity bi grams for finding out important entities motivated by [\ref{relgram}].

\section{Applications}
We list a few applications of the sense prior and outline an application of the entity bigrams.

 \subsection{Sense Prior}
 A prior over the sense will be helpful in many applications related to information retrieval. 
 \begin{itemize}
  \item Entity Querying
  \item Knowledge graph based searching
 \end{itemize}

 \subsection{Entity Bigrams}
 
 Given an entity, we want to find out other important entities that are related to it.
 For example, given an entity \textbf{Barack Obama, President of the USA}, we need to provide top 10 entities that are
 ``close'' to Barack Obama the President. Since the solution is only a slight modification of the solution 
 presented in [\ref{cohschemas}] for finding out important relations, we only sketch an outline here. 
 
For the entity we are interested in, Say X, create a node. Now attach to the node X all the entities E for which
$P(E|X) >  \epsilon$ where $\epsilon$ is some threshold. Let the weight of the edge be defined as

\begin{equation}
\tag{3}
 P(E|X) + P(X|E) 
\end{equation}

We then apply personalized page rank on the X sub graph, starting with X having a 
page rank of 1 and other nodes having a page rank of 0. We can then sort the nodes
based on the their page ranks upon convergence. 
 


\section{Baseline Approach}
How do we collect the aforementioned statistics?
This question shouldn't be too difficult to answer now. The whole of part 3
was dedicated towards tagging entity mentions in the text. We can use any of the 
methods (for example, AIDA can be set up as a rest service) to tag the corpus, and then iterate over the corpus to collect these statistics
in single pass. 
\section{Solution based on estimating class ratios}
While estimating class ratios by doing per mention disambiguation seems pretty intuitive, we are doing more than what we need to do.
Intuitively, we are not interested in what each mention disambiguates to, a count of how many times does a particular entity appears
is the desideratum. Recently, there has been a progress on methods for predicting the class ratio directly, without going through the 
label and collect route. In particular, [\ref{mmd}] discuss a solution based on maximum mean discrepancy and proves some upper bounds 
on errors. 

If mmd really works, we should expect better estimation of the sense prior and the entity grams. The next chapter outlines the mmd based 
solution and how mmd may be used to estimate the sense priors for different entities. 

\chapter{MMD for estimating ratios of named entities in text}

\section{Introduction}

\section{MMD Formulation}


\chapter{Conclusion and further readings}
TODO

For this report, only a small subset of the papers was selected to cover as much ground as possible.
The following list may be valuable to the interested readers.

\begin{itemize}
 \item \textbf{We have emphasized on Wikipedia as the catalog. The following work presents a general approach} \\
 Sil, Avirup, et al. "Linking named entities to any database." Proceedings of the 2012 Joint Conference on Empirical Methods in Natural Language Processing and Computational Natural Language Learning. Association for Computational Linguistics, 2012.
 \item \textbf{Large scale named entity disambiguation.} \\
 Cucerzan, Silviu. ``Large-Scale Named Entity Disambiguation Based on Wikipedia Data.'' EMNLP-CoNLL. Vol. 7. 2007.
 \item \textbf{One of the initial works on NED} \\
 Bunescu, Razvan C., and Marius Pasca. "Using Encyclopedic Knowledge for Named entity Disambiguation." EACL. Vol. 6. 2006.
 \item \textbf{Yago, a clean catalog of Wikipedia entities and the relations } \\
 Suchanek, Fabian M., Gjergji Kasneci, and Gerhard Weikum. "Yago: a core of semantic knowledge." Proceedings of the 16th international conference on World Wide Web. ACM, 2007.
 \item \textbf{Quick entity annotations for short text} \\
 Suchanek, Fabian M., Gjergji Kasneci, and Gerhard Weikum. "Yago: a core of semantic knowledge." Proceedings of the 16th international conference on World Wide Web. ACM, 2007.
\end{itemize}


\begin{thebibliography}{9}
 \bibitem{A} \label{snowball}
 http://dl.acm.org/citation.cfm?id=336644Agichtein, Eugene, and Luis Gravano. "Snowball: Extracting relations from large plain-text collections." Proceedings of the fifth ACM conference on Digital libraries. ACM, 2000.
\bibitem{B} \label{mintz}
Mintz, Mike, et al. "Distant supervision for relation extraction without labeled data." Proceedings of the Joint Conference of the 47th Annual Meeting of the ACL and the 4th International Joint Conference on Natural Language Processing of the AFNLP: Volume 2-Volume 2. Association for Computational Linguistics, 2009.
 \end{thebibliography}

\end{document}          
