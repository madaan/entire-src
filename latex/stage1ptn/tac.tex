\documentclass{beamer}

\usepackage[utf8]{inputenc}
\usepackage{default}
\usetheme{Warsaw}

\begin{document}

\begin{frame}{Cold Start Knowledge Base Population at 2014}

\begin{itemize}
 
 \item Knowledge Base Population (KBP) track of TAC encourages the development of systems that can match entities mentioned in natural texts with those
appearing in a knowledge base and extract novel information about entities from a document collection and add it to a new or existing knowledge base. \pause
 
 \item Some example relations:
    \begin{itemize}
      \item children of
      \item city of birth
      \item shareholders
      \item countries of residence
    \end{itemize}
\end{itemize}

\end{frame}

\begin{frame}

\begin{itemize}
  \item  Modeled the problem using distant supervision
  \item Used Freebase as an existing Knowledge base.
    
    \item \textbf{Freebase: }Freebase is a large collaborative knowledge base consisting of metadata composed mainly by its community members. 
    \item It is an online collection of structured data harvested from many sources, including individual, user-submitted wiki contributions.
 \end{itemize}

 
\end{frame}

\begin{frame}{Corpus}
 \begin{itemize}
  \item The TAC corpus consisted of three type of documents: \pause
    \begin{itemize}
	\item \textbf{discussion forums } 99,063 English discussion forum documents selected from the BOLT Phase 1 discussion forums source data releases. Each forum includes at least 5 posts. \pause

	\item \textbf{newswire } 1,000,257 documents selected from English Gigaword Fifth Edition. \pause

	 \item \textbf{web} 999,999 English web documents selected from various GALE web collections. \pause
    \end{itemize}
    
      \item We have submitted the knowledge base populated using our techniques on the above corpus and waiting for results.
 \end{itemize}


 
\end{frame}


\end{document}
